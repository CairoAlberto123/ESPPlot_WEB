% Código desenvolvido por: Cairo Alberto
% Estudante de Engenharia da Computação - PUC Goiás
% Repositório: https://github.com/CairoAlberto123
% Data de criação: 2025-04-01
% Descrição: Documento de documentação técnica do Sistema Integrado para Leitura de ADC via ESP32 com Servidor Flask e Interface Web
% Tecnologias utilizadas: ESP32, Arduino IDE, Python, Flask, HTML, CSS, JavaScript, LaTeX
% Licença: MIT
% 2025 Cairo Alberto - Todos os direitos reservados.

\documentclass[12pt,a4paper]{article}
\usepackage[utf8]{inputenc}
\usepackage[brazil]{babel}
\usepackage{indentfirst}
\usepackage{graphicx}
\usepackage{hyperref}
\usepackage{amsmath}
\usepackage{geometry}
\usepackage{fancyhdr}
\usepackage{cite}

\geometry{left=3cm,right=2cm,top=3cm,bottom=2cm}
\pagestyle{fancy}
\fancyhead[L]{Sistema Integrado ADC}
\fancyhead[R]{Cairo Alberto}
\fancyfoot[C]{\thepage}

\begin{document}

\begin{titlepage}
    \centering
    \vspace*{2cm}
    {\LARGE \textbf{Sistema Integrado para Leitura de ADC via ESP32 com Servidor Flask e Interface Web}\par}
    \vspace{2cm}
    {\Large Documento de Documentação Técnica\par}
    \vspace{2cm}
    {\large Cairo Alberto \par}
    {\large \texttt{https://github.com/CairoAlberto123} \par}
    \vfill
    {\large 2025 Cairo Alberto - Todos os direitos reservados.\par}
\end{titlepage}

\tableofcontents
\newpage

\section{Introdução}
Este documento tem como objetivo descrever detalhadamente o funcionamento e a implementação do sistema integrado para leitura de ADC utilizando ESP32, servidor Flask e interface web. A documentação segue as normas da ABNT, apresentando referências, descrições dos componentes e manuais de funcionamento.

\section{Firmware para ESP32}
\subsection{Descrição Geral}
O firmware desenvolvido para o ESP32 realiza a leitura do ADC e envia os valores lidos via comunicação serial USB. O código foi implementado na IDE do Arduino, com estrutura modular e comentários que facilitam a manutenção e a extensão do sistema.

\subsection{Conexões}
\begin{itemize}
    \item \textbf{ADC:} Conectar os sinais analógicos (+ e -) conforme a especificação do sensor utilizado.
    \item \textbf{USB:} O conector USB-C é utilizado para programação e comunicação serial.
\end{itemize}

\subsection{Manual de Funcionamento}
O firmware inicia a comunicação serial a 115200 bps, realiza leituras periódicas do ADC (pino 34, por exemplo) e transmite os valores lidos. Funções específicas para inicialização e leitura do ADC foram implementadas, permitindo fácil adaptação a diferentes sensores ou requisitos de projeto.

\section{Servidor Flask em Python}
\subsection{Descrição Geral}
O servidor desenvolvido em Python utiliza o framework Flask para monitorar a porta serial e coletar os dados enviados pelo ESP32. Além da leitura contínua, o servidor implementa filtros (passa-baixa e passa-alta), permite a gravação dos dados e gera um vídeo com os gráficos para posterior download.

\subsection{Bibliotecas Utilizadas}
As principais bibliotecas utilizadas são:
\begin{itemize}
    \item \textbf{Flask} - Para o desenvolvimento do servidor web;
    \item \textbf{PySerial} - Para a comunicação serial com o ESP32;
    \item \textbf{NumPy} - Para o processamento numérico dos dados;
    \item \textbf{Matplotlib} - Para a geração de gráficos;
    \item \textbf{OpenCV} - Para a criação de vídeos a partir dos gráficos.
\end{itemize}

\subsection{Arquivo requirements.txt}
\begin{verbatim}
Flask
pyserial
numpy
matplotlib
opencv-python
\end{verbatim}

\subsection{Manual de Funcionamento}
O servidor realiza a leitura contínua dos dados via porta serial, aplica os filtros configurados e envia os dados para a interface web para visualização em tempo real. A funcionalidade de gravação permite salvar os dados brutos em um arquivo de texto e gerar um vídeo com o gráfico dos dados capturados.

\section{Interface Web}
\subsection{Descrição Geral}
A interface web foi desenvolvida com HTML, CSS e JavaScript, proporcionando:
\begin{itemize}
    \item Seleção da porta serial para conexão com o ESP32;
    \item Configuração dos filtros (passa-baixa e passa-alta);
    \item Visualização dos dados em tempo real através de um gráfico dinâmico;
    \item Controle de gravação dos dados, com indicadores visuais e temporizador.
\end{itemize}

\subsection{Manual de Uso}
O usuário pode selecionar a porta serial desejada, configurar os filtros e iniciar/parar a gravação dos dados. Os dados são atualizados dinamicamente no gráfico, e ao finalizar a gravação, os arquivos de dados e vídeo ficam disponíveis para download.

\section{Conclusão}
O sistema integrado apresentado permite a captura, processamento e visualização dos dados do ADC de forma modular e expansível. A documentação técnica e o código comentado facilitam a manutenção e futuras atualizações, atendendo aos requisitos propostos e seguindo as normas da ABNT.

\section{Referências}
\begin{thebibliography}{9}
\bibitem{Arduino}
Arduino. \textit{Arduino Reference}. Disponível em: \url{https://www.arduino.cc/reference/en/}. Acesso em: [data de acesso].
\bibitem{Flask}
Flask. \textit{Flask Documentation}. Disponível em: \url{https://flask.palletsprojects.com/}. Acesso em: [data de acesso].
\bibitem{PySerial}
PySerial. \textit{PySerial Documentation}. Disponível em: \url{https://pythonhosted.org/pyserial/}. Acesso em: [data de acesso].
\bibitem{Matplotlib}
Matplotlib. \textit{Matplotlib Documentation}. Disponível em: \url{https://matplotlib.org/}. Acesso em: [data de acesso].
\bibitem{OpenCV}
OpenCV. \textit{OpenCV Documentation}. Disponível em: \url{https://docs.opencv.org/}. Acesso em: [data de acesso].
\end{thebibliography}

\end{document}
