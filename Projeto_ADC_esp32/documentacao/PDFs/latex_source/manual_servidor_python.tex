\documentclass[12pt,a4paper]{article}
\usepackage[utf8]{inputenc}
\usepackage[lmargin=3cm,rmargin=2cm,tmargin=3cm,bmargin=2cm]{geometry}
\usepackage[brazil]{babel}
\usepackage{hyperref}
\usepackage{listings}

\title{Manual do Servidor Flask}
\author{Cairo Alberto}
\date{\today}

\begin{document}

\maketitle

\section{Requisitos do Sistema}
\begin{itemize}
\item Python 3.8+
\item Dependências (requirements.txt):
\begin{verbatim}
flask==2.0.1
flask-socketio==5.1.1
pyserial==3.5
numpy==1.21.2
scipy==1.7.1
\end{verbatim}
\end{itemize}

\section{Arquitetura do Sistema}
\begin{itemize}
\item Comunicação Serial: Thread dedicada
\item WebSockets: Atualizações em tempo real
\item Filtros Digitais: Implementados com SciPy
\end{itemize}

\section{API Endpoints}
\begin{itemize}
\item \textbf{GET /}: Interface web principal
\item \textbf{WebSocket /socket.io}: Comunicação bidirecional
\end{itemize}

\section{Configuração de Filtros}
Fórmula do filtro passa-baixa:
\[
H(s) = \frac{1}{\sqrt{1 + \left(\frac{f}{f_c}\right)^{2n}}}
\]
Onde:
\begin{itemize}
\item \(f_c\): Frequência de corte
\item \(n\): Ordem do filtro (padrão=5)
\end{itemize}

\end{document}